\documentclass[11pt,titlepage]{article}

\usepackage[spanish,activeacute]{babel}
%\usepackage[latin1]{inputenc}
\usepackage{amsmath,amsthm}	% El segundo es para entorno demostración
\usepackage{amssymb,latexsym}
\usepackage[nomessages]{fp} % For doing simple calculations.

\decimalpoint		% Use punto en vez de coma, asi: 5.45
\allowdisplaybreaks	% Cambios de pagina en medio de alineaciones en ecuaciones, ponerlo siempre
\newcommand{\tto}{\longrightarrow}
\newcommand{\N}{{\ensuremath{\mathbb{N}}}}
\newcommand{\Q}{{\ensuremath{\mathbb{Q}}}}
\newcommand{\R}{{\ensuremath{\mathbb{R}}}}
\newcommand{\Z}{{\ensuremath{\mathbb{Z}}}}
\newcommand{\C}{{\ensuremath{\mathbb{C}}}}
\newcommand{\parcial}[2]{\frac{\partial #1}{\partial #2}}
\newcommand{\upla}[2]{(#1_1,#1_2,\ldots,#1_{#2})}
\newcommand{\kupla}[3][k]{(#2_{#3},\,\ldots,#2_{#1})}

% Teoremas y estructuras similares
%\swapnumbers
\theoremstyle{plain}
	\newtheorem{prop}{Proposici\'{o}n}[section]
	\newtheorem{teor}[prop]{Teorema}
	\newtheorem{coro}[prop]{Corolario}
	\newtheorem{lema}[prop]{Lema}
\theoremstyle{defintion}
	\newtheorem{defi}{Definici\'{o}n}[section]
	\newtheorem{ejem}{Ejemplo}
	\newtheorem{ejer}{Ejercicio}[section]
\theoremstyle{remark}
	\newtheorem{nota}{Nota}
	\newtheorem*{notac}{Notaci\'{o}n}



\title{Veni, vidi, vici\\\small{Sieg Heil Sublime}}
\author{Mauricio Mejia\\Mario Mejia\thanks{Hola mundo}}

\begin{document}

\maketitle
\section{Matem'aticas}
La siguiente f'ormula: $f(x,\ y)=ax+by$ es una fórmula no desplegada, mientras que:
\[f(x,y)=ax+by.\]
si lo est'a, esbastante {\it cool}, ¿no?
\\
Para todo $\varepsilon>0$ existe una $\delta>0$ tal que $0<|x-c|<\delta$ implica que $|f(x)-L|<\varepsilon$.
\[
\text{texto dentro de la fórmula, wieeeerd\ }
X_i \preccurlyeq X_j
\]
Los siguientes s'imbolos pertenecen al paquete {\tt latexsym}: $\ulcorner$, $\urcorner$, $\llcorner$, $\lrcorner$.
Ahgamos algunas sumas:
\[
\sum_{i=0}^{NM} \beta_i
\]
Otra f'ormula m'as complicada:
\[
\sum_{\substack{j=k\\i=k-1}}^{n,\ m}c_{i}\alpha_{j}
\]
Delimitadores:
\[
\left( \frac{1-n}{1+n} \right)^{n+1}
\]
\[
\left. \frac{dy}{dx} \right|_{x=b}=b+1
\]
\[
\biggl( \frac{1-n}{1+n} \biggr)^{n+1}
\]
Los siguientes son casos!!!:
\[f_n(x)=
	\begin{cases}
	-x^{2}+n,	&	\text{si $x<0$ y $n$ es par},\\
	\alpha +x,	&	\text{si $x>0$},\\
	x^{2},		&	\text{en otros casos.}
	\end{cases}
\]
La definici'on formal de integral de Riemann:
\[
\int_a^b f(x)\,dx=
\lim_{|P|\to 0}\,\sum_{i=0}^{n} f(\overline{x_i})\,\Delta_ix
\]
M'as integrales:
\[
\oint\limits_{(0,0)}^{(1,1)} f\cdot d\alpha
\]
Que tal esto:
\[
X\:\underrightarrow{\text{x+y}}\:Y
\]
Hagamos algunas matrices:
\[
	\begin{bmatrix}
	0 &  i & -i &  1\\
	1 &  0 &  i & -1\\
	i & -1 &  0 & -i
	\end{bmatrix}
	\begin{bmatrix}
	x_1 \\ x_2 \\ x_3 \\ x_4
	\end{bmatrix}
\]
Matrices con puntos:
\[
	\begin{bmatrix}
	a_{11} & a_{12} & \cdots & a_{1n}\\
	a_{21} & a_{22} & \cdots & a_{2n}\\
	\hdotsfor{4}\\
	a_{m1} & a_{m2} & \cdots & a_{mn}
	\end{bmatrix}
	\begin{bmatrix}
	a_{11} & a_{12} & \cdots & a_{1n}\\
	a_{21} & a_{22} & \cdots & a_{2n}\\
	\vdots & \vdots & \ddots & \vdots\\
	a_{m1} & a_{m2} & \cdots & a_{mn}
	\end{bmatrix}
\]
Sea $\mathfrak{F}$ una $\sigma$-'algebra definida sobre $\Omega$.
Esta es una de las que salen:
\[
\overset{a}{\underset{b}{W}}
\]
\[
X \overset{*}{\backsim} Y
\]
\[
X \overset{\text{ind}}{=} Y
\]

Sea \R\ el conjunto de todos los reales y \C\ el de los complejos y sea $f:\R\tto\C$ una funci'on.

\[
\nabla f(x,\,y)=\left(\parcial{f}{y},\,\parcial{f}{y}\right)
\]

\[\upla{a}{n}\]
\[\kupla[m]{x}{2}\]
\par 
Las rosas son rojas y el sol es amarillo, ¿qui'en tiene el poder para desafiar a los grandes ejercitos de Mordor y Isengard? Todo est'a realemente jodido.
\par 
El mar es azul, las rosas son rojas.
\section{La teor'ia de Galois}
\begin{align*}
a*(a'*b) &= (a*a')*b & &\text{por la ley asociativa}\\
		 &= c*b & &\text{por la definici'on de $a'$}\\
		 &{=}_{e} b & &\text{por ser $e$ elemento identidad}
\end{align*}

\begin{defi}[Extensi\'{o}n de Galois]
Una extensi'on finita, normal y separable $E$ de un campo $F$ se llama una {\bf extensi'on de Galois} de $F$.
\end{defi}
\par El siguiete resultado es el llamado teorema fundamental de la teor'ia de Galois.
\begin{teor}[Teorema de Galois]
Sea $E$ una extensi'on de Galois de $F$ y sea $K$ una campo tal que $F\subseteq K\subseteq E$.
Entonces $K\mapsto G(E/K)$ establece una correspondencia biyectiva entre el conjunto de los subconjuntos de $E$ que contienen a $F$ y los subgrupos de $G(E/F)$.
\end{teor}

\begin{ejer}
Demuestre la existencia de Dios.
\end{ejer}

\begin{teor}
Dios existe
\end{teor}
\begin{proof}[Demostraci\'{o}n:]
Procediendo por contradicci'on, supongamos que Dios no existe
\[ \sec^2(x)=1+\tan^2(x). \qedhere \]

\end{proof}


\end{document}
