\documentclass[11pt]{article}

\usepackage[spanish,activeacute]{babel}
\title{Acerca de tablas en \latex}
\author{El de siempre}

\begin{document}
\par
Esta es la primera tabla, un deleite de encanto y sencillez.
\begin{center}
\begin{tabular}{|l|l|c|}\hline
	\multicolumn{3}{|c|}{PART'ICULAS AT'OMICAS ELEMENTALES}\\ \hline\hline
	\sf{Part'icula:} & \sf{Descubridor:} & \sf{A'no de descubrimiento:}\\ \hline\hline
	Electr'on & Joseph J. Thomson & 1887\\ \hline
	Prot'on & James Rutherford & 1919\\ \hline
	Neutrón & James Chadwick & 1932\\ \hline
	Positr'on & Carl D. Anderson & 1932\\ \hline
\end{tabular}
\end{center}
\par 
A continuación tenemos un esperpento que podr'ia resultar 'util en muchas situaciones de la vida acad'emica.
\begin{center}
\begin{tabular}{*{4}{|c}|}\hline
	Uno & Dos & Tres & Cuatro\\ \hline
	&&A&B \\ \cline{3-4}
	&&C& \\ \cline{1-3}
	&D&& \\ \cline{2-2}
	&E&& \\ \hline
	1&2&3&4 \\ \hline
\end{tabular}
\end{center}

\end{document}