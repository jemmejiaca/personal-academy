\documentclass{article}
\usepackage{pst-all}
\usepackage{pstcol}
\usepackage{pstricks}


\newpsobject{grilla}{psgrid}{subgriddiv=1,griddots=10,gridlabels=6pt}

\begin{document}
\begin{center}
	\psset{unit=3.5cm}
	\begin{pspicture}(-1.5,-1)(1.5,0)\grilla
		\pstextpath[c]{\pscurve(-1.41,0)(-1,-1)(0,0)(1,-1)(1.41,0)}
		{\color{blue}\large Esta frase sinuosa y peculiar est\'a escrita a lo largo de la curva $y=x^4-2x^2$}
	\end{pspicture}
\end{center}

\par He aqui otra maginfica grilla:
\begin{center}
	\psset{unit=0.5cm}
	\begin{pspicture}(-5,0)(5,4)\grilla
	
	\end{pspicture}
\end{center}

%{\grey \large Esta frase est\'a escrita en color gris claro}

\par El comando {\tt \textbackslash psline} sirve para unir para unir con segmentos de recta los puntos  $(x_0,\,y_0)$, $(x_1,\,y_1)$, $\ldots$, $(x_n,\,y_n)$. Por ejemplo:

\begin{center}
	\begin{pspicture}(0,0)(6,4)\grilla
		\psline[linewidth=1.5pt,linearc=0.1]{>>->>}(0,4)(6,0)(0,0)(4.5,2.5)
	\end{pspicture}
\end{center}

\par Aqui va otra figura:
\begin{center}
	\psset{unit=0.8cm}
	\begin{pspicture}(0,0)(5,5) %\grilla
		\psline{->}(0,1)(5,1)
		\psline{->}(1,0)(1,5)
		\psline[linewidth=2pt]{[-]}(2,1)(4.5,1)
		\psline[linewidth=2pt]{(-)}(1,1.5)(1,4)
	\end{pspicture}
\end{center}

\par Este es un intento de tri\'angulo hecho por mi:
\begin{center}
	\psset{unit=0.5cm}
	\begin{pspicture}(0,0)(16,16) \grilla
		\pspolygon[linewidth=1.5pt](2,2)(7,8)(7,2)
		\pswedge[linewidth=1.5pt](2,2){1}{0}{50.19}
	\end{pspicture}
\end{center}

\par Una par\'abola 3D:
\begin{center}
	\begin{pspicture}(-4,-4)(6,10)\grilla
		\parabola[gradangle=90,fillstyle=gradient,gradmidpoint=12](-3,7)(0,1)
		\psellipse[fillstyle=gradient](0,7)(3,0.5)
		\psline{->}(-4,0)(6,0)
		\psline{->}(4,4)(-3,-3)
		\psline{->}(0,6.5)(0,10)
		\psline(0,-3)(0,1)
	\end{pspicture}
\end{center}

\end{document}